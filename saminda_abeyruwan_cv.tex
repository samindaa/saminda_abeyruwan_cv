%% start of file `template.tex'.
%% Copyright 2006-2013 Xavier Danaux (xdanaux@gmail.com).
%
% This work may be distributed and/or modified under the
% conditions of the LaTeX Project Public License version 1.3c,
% available at http://www.latex-project.org/lppl/.


\documentclass[11pt,a4paper,roman]{moderncv}        % possible options include font size ('10pt', '11pt' and '12pt'), paper size ('a4paper', 'letterpaper', 'a5paper', 'legalpaper', 'executivepaper' and 'landscape') and font family ('sans' and 'roman')

% moderncv themes
\moderncvstyle{classic}                             % style options are 'casual' (default), 'classic', 'oldstyle' and 'banking'
\moderncvcolor{blue}                               % color options 'blue' (default), 'orange', 'green', 'red', 'purple', 'grey' and 'black'
%\renewcommand{\familydefault}{\sfdefault}         % to set the default font; use '\sfdefault' for the default sans serif font, '\rmdefault' for the default roman one, or any tex font name
%\nopagenumbers{}                                  % uncomment to suppress automatic page numbering for CVs longer than one page

% character encoding
%\usepackage[utf8]{inputenc}                       % if you are not using xelatex ou lualatex, replace by the encoding you are using
%\usepackage{CJKutf8}                              % if you need to use CJK to typeset your resume in Chinese, Japanese or Korean

\usepackage{paralist}
\usepackage[resetlabels]{multibib}

% Multibib citation definitions
\newcites{journal,conference,other}{Refereed Journals, Refereed Conference and Workshop 
Proceedings, 
Non-Refereed Scientific Contributions} 

% adjust the page margins
\usepackage[scale=0.75]{geometry}
%\setlength{\hintscolumnwidth}{3cm}                % if you want to change the width of the column with the dates
%\setlength{\makecvtitlenamewidth}{10cm}           % for the 'classic' style, if you want to force the width allocated to your name and avoid line breaks. be careful though, the length is normally calculated to avoid any overlap with your personal info; use this at your own typographical risks...

% personal data
\name{Saminda}{Abeyruwan}
\title{Curriculum vitae}                               % optional, remove / comment the line if not wanted
%\address{street and number}{postcode city}{country}% optional, remove / comment the line if not wanted; the "postcode city" and "country" arguments can be omitted or provided empty
\phone[mobile]{+1~(305)~457~9753}                   % optional, remove / comment the line if not wanted; the optional "type" of the phone can be "mobile" (default), "fixed" or "fax"
%\phone[fixed]{+2~(345)~678~901}
%\phone[fax]{+3~(456)~789~012}
\email{samindaa@gmail.com}                               % optional, remove / comment the line if not wanted
\homepage{www.saminda.org}                         % optional, remove / comment the line if not wanted
%\social[linkedin]{john.doe}                        % optional, remove / comment the line if not wanted
%\social[twitter]{jdoe}                             % optional, remove / comment the line if not wanted
%\social[github]{jdoe}                              % optional, remove / comment the line if not wanted
%\extrainfo{additional information}                 % optional, remove / comment the line if not wanted
%\photo[64pt][0.4pt]{picture}                       % optional, remove / comment the line if not wanted; '64pt' is the height the picture must be resized to, 0.4pt is the thickness of the frame around it (put it to 0pt for no frame) and 'picture' is the name of the picture file
%\quote{Some quote}                                 % optional, remove / comment the line if not wanted

% to show numerical labels in the bibliography (default is to show no labels); only useful if you make citations in your resume
%\makeatletter
%\renewcommand*{\bibliographyitemlabel}{\@biblabel{\arabic{enumiv}}}
%\makeatother
%\renewcommand*{\bibliographyitemlabel}{[\arabic{enumiv}]}% CONSIDER REPLACING THE ABOVE BY THIS

% bibliography with mutiple entries
%\usepackage{multibib}
%\newcites{book,misc}{{Books},{Others}}
%----------------------------------------------------------------------------------
%            content
%----------------------------------------------------------------------------------

%\renewcommand*{\namefont}{\fontsize{38}{40}\mdseries\upshape}
\renewcommand*\namefont{\fontsize{25}{18}\selectfont}
% \renewcommand*\titlefont{\fontfamily{pzc}\fontsize{20}{24}\selectfont}
% \renewcommand*\addressfont{\fontfamily{pzc}\selectfont}
% \renewcommand*\sectionfont{\fontfamily{pzc}\fontsize{20}{24}\selectfont}

\begin{document}
%\begin{CJK*}{UTF8}{gbsn}                          % to typeset your resume in Chinese using CJK
%-----       resume       ---------------------------------------------------------
\makecvtitle

\section{Research Interests}
\cvitem{~}{I am an enthusiastic computer scientist who pursues to solve research problems in 
artificial intelligence, machine learning, reinforcement learning, multi-agent systems, robotics, 
embedded systems, semantic web, learnable knowledge representations, and real-time reasoning. \newline 
I am 
specifically focused on the problem of closing the gap between conceptual knowledge representation 
formalisms and the low-level sensorimotor data, and then generalize the knowledge for unseen 
circumstances. I favor team work and I also have worked in interdisciplinary research projects with 
domain experts from other research ares, which expanded my research horizons and analytical skills.}

\section{Education}
\cventry{Aug~2010--Aug~2015}{PhD}{Computer Science, University of Miami, Coral Gables, Florida}{}{USA}{Advisor: Dr. Ubbo Visser \\ Thesis: Learnable Knowledge for Autonomous Agents \\ GPA: 3.98}  % arguments 3 to 6 can be left empty
\cventry{Aug~2008--May~2010}{MS}{Computer Science, University of Miami, Coral Gables, Florida}{}{USA}{Advisor: Dr. Ubbo Visser \\ Thesis: PrOntoLearn: Unsupervised Lexico-Semantic Ontology Generation using Probabilistic Methods \\ GPA: 4.0}  % arguments 3 to 6 can be left empty
\cventry{Jan~2001--Nov~2004}{B.Sc.}{Electrical Engineering, University of Moratuwa, Moratuwa}{}{Sri Lanka}{Project: Micro Scale Village Based Dendro Power Development \\ First class honors in Electrical Engineering. GPA: 3.79 -- Ranked 1$^{\mbox{st}}$ out of 51 }  % arguments 3 to 6 can be left empty


% Publications from a BibTeX file without multibib
%  for numerical labels: \renewcommand{\bibliographyitemlabel}{\@biblabel{\arabic{enumiv}}}% CONSIDER MERGING WITH PREAMBLE PART
%  to redefine the heading string ("Publications"): \renewcommand{\refname}{Articles}

%\nocite{*}
%\bibliographystyle{ieeetr}
%\bibliography{saminda_abeyruwan_publications}                        % 'publications' is the name of a BibTeX file

\section{Publications}
% Journal articles
\bibliographystylejournal{ieeetr}
\nocitejournal{*}
\bibliographyjournal{journal}

\vspace{2mm}
%
% Conference articles
\bibliographystyleconference{ieeetr}
\nociteconference{*}
\bibliographyconference{conference}

\vspace{2mm}
%
% Other scientific
\bibliographystyleother{ieeetr}
\nociteother{*}
\bibliographyother{other}


%\section{Awards}

%\section{Master thesis}
%\cvitem{title}{\emph{Title}}
%\cvitem{supervisors}{Supervisors}
%\cvitem{description}{Short thesis abstract}

%\section{Experience}
\section{Academic Research}
\cventry{Aug~2010--Jul~2015}{Research Assistant}{University of Miami, Department of Computer Science, 
Coral Gables}{}{USA}{\begin{enumerate}\item \textbf{BioAssay Ontology Development} 
\begin{itemize}\item During this project, we have used Description Logic and Bayesian networks to 
capture the semantics 
in high throughput screening assays for minimal information formalisms and assay annotation. \item 
I have developed a state-of-the art ontology modularization framework to handle large scale 
ontologies without compromising the interpretation of the domain of discourse. \item I have 
developed a framework to reason large scale ontologies using standard OWL 2 reasoners and the 
Hadoop Map-Reduce framework. This framework has been used in the preliminary stages of the 
knowledge-based reporting project.   \end{itemize} \item \textbf{RegenBase Ontology Development} 
\begin{itemize}\item The development of  OWL 2 ontologies that allow domain experts to link data 
and results from studies on nervous system injury and disease to data and knowledge from other 
domains with an emphasis on molecular targets and the small molecules that perturb their function 
to speed the development of novel therapeutics. \item I have developed  various Prot\'{e}g\'{e} 
plugins (e.g. paper annotation plugin, OBO2OWL) based on the requirements from domain experts to 
expedite knowledge 
acquisition, and assay annotation phases. These tools are used in related ontology projects in different capacities.  \end{itemize} \item \textbf{RoboCup Soccer} \begin{itemize}\item I have used the RoboCup Standard Platform (NAO humanoid robots) and the 3D Soccer Simulation League as test environments to conduct research on learnable knowledge representations and real-time reasoning to build autonomous agents. \item I have extended the $\mathcal{SROIQV}^\mathcal{D}$ description logic to represent dynamic  knowledge  of  the world without compromising satisfiability guarantees. In addition to this, I have used general value function in reinforcement learning as a method to represent predictive knowledge in role assignments in soccer formations.     \end{itemize} \item \textbf{Embedded Systems} \begin{itemize} 
\item  I have developed  unification methodologies to detect abnormal events, e.g., a fall event, for humans and biped humanoid robots while performing normal activities such as jogging, running and so forth. \item In this study, I have developed: \begin{inparaenum}[1)] \item 
methods to learn and predict different activities for humans and robots; and \item software 
tools to realize these functions  on embedded devices. \end{inparaenum} The main contributions include: 
\begin{inparaenum}[1)] \item detection of falls for both humans and robots within a unified 
framework; and \item  a  novel software development environment for embedded systems. 
\end{inparaenum}
\end{itemize} \end{enumerate}}

\section{Teaching}
\cventry{Fall~2011}{Instructor}{University of Miami, Department of Computer Science, Coral Gables}{}{USA}{\begin{itemize} \item I have prepared and taught the graduate course "Semantic Web" with Dr. Ubbo Visser. I have prepared lecture materials, assignments, and course schedule from scratch. \end{itemize}}
\cventry{Aug~2008--May~2010}{Teaching Assistant}{University of Miami, Department of Computer Science, Coral 
Gables}{}{USA}{\begin{itemize} \item I have taught and graded computer science courses during the 
fall and spring semesters for the following classes: \begin{enumerate}
\item Computer Programming I (CSC120) \item Computer Programming II (CSC220) \item  
Computer Organization and Architecture (CSC314); \item C Programming and UNIX (CSC322) \item   
Principles of Computer Operating Systems (CSC421) and \item Introduction to Artificial 
Intelligence (CSC545) \end{enumerate} I was responsible for teaching labs, helping with 
assignments, and giving occasional seminars. I received  awards in recognition of my teaching 
skills and performance from the department. 
\end{itemize}}


\section{Employment}
\cventry{Aug~2015-- }{Software Engineer IV}{Cisco Systems, Inc.}{}{USA}{\begin{itemize} \item Scalable and distributed software development engineer for network controller.
		\item ACI APIC controller engineer.  \end{itemize} }
\cventry{Summer~2014}{Software Developer}{University of 
Miami, Center for Computational Science, Miami}{}{USA}{\begin{itemize} \item The development of 
internal tools to support research activities in different projects, specially, BioAssay Ontology 
and RegenBase projects. \item Developed a Hadoop based framework to support knowledge 
base reporting and parallel reasoning for OWL ontologies. \end{itemize} }
\cventry{Summer~2013}{Software Developer}{University of 
Miami, Center for Computational Science, Miami}{}{USA}{}
\cventry{Summer~2012}{Software Developer}{University of 
Miami, Center for Computational Science, Miami}{}{USA}{}
\cventry{Summer~2011}{Software Developer}{University of 
Miami, Center for Computational Science, Miami}{}{USA}{}
\cventry{Nov~2005--Aug~2008}{Technical Lead}{WSO2 Inc, Colombo}{}{Sri Lanka}{\begin{itemize}
 \item The design and development of  middleware applications using the principles of 
 service-oriented architecture and Apache Web services projects. These applications were 
 implemented in C\texttt{++} ~and Java languages to integrate services with Web 2.0 front ends.
 \item The research and development of applications that utilized OSGi technologies in server-side, and embedded systems.
 \item Managed a software team to integrate the Web Services Application Server with Apache Web services. 
 \item The development of enterprise service bus with Apache Synapse project.
\end{itemize}}
%\cventry{Spring~2003}{Software Development Intern}{Ceylon Electricity Board, Colombo}{}{Sri Lanka}{\begin{itemize} \item Assisted in the development of a project based on C++ that monitors the national power grids.\end{itemize}}


\section{Community}
\cventry{Nov~2005-- }{Committer and Program Committee Member}{Apache Software 
Foundation}{}{}{\begin{itemize} \item I am an active committer for Apache Axis2, Synapse, and Axiom 
projects. I am a founding member of Apache Synapse project. \end{itemize}}

\section{Invited Talks}
%\cventry{Nov~2010}{URSW, 2010}{}{``PrOntoLearn: Unsupervised Lexico-Semantic Ontology Generation using Probabilistic Methods''}{}{}
%\cventry{May~2013}{ALA, 2013}{}{``Dynamic Role Assignment using General Value Functions''}{}{}
%\cventry{July~2014}{RoboCup, 2014}{}{``A New Real-Time Algorithm to Extend DL Assertional Formalism to Represent and Deduce Entities in Robotic Soccer''}{}{}
\cventry{Dec~2014}{WSO2, Inc}{Colombo, Sri Lanka}{``Learnable Knowledge Representations for 
Autonomous Agents''}{}{}{}


%\section{Poster Presentations}
%\cventry{Nov~2010}{ISWC, 2010}{}{``BAOSearch: A Semantic Web Application for Biological Screening and Drug Discovery Research''}{}{}


\section{Services}
\cventry{}{Reviewer}{}{}{}{\begin{itemize} 
\item International Journal of Advanced Robotic Systems.
\item International Semantic Web Conference.
\item Journal of Intelligent \& Robotic Systems.
\item Journal on Multimodal User Interfaces.
\item Robotics -- Open Access Journal.
\item RoboCup Symposium Proceedings.
\item Uncertainty Reasoning for the Semantic Web II.6
\end{itemize}}

\section{Memberships}
\cvitem{}{Student members of ACM, IEEE, and IAAA.}



\section{Software}
\cventry{Jan~2013--}{RLLib}{C\texttt{++} Template Library to Predict, Control,  Learn Behaviors, and Represent Learnable Knowledge using On/Off Policy Reinforcement Learning}{}{}{\begin{itemize} 
\item This is a lightweight C\texttt{++} template library that implements incremental, standard, and gradient temporal-difference learning algorithms in reinforcement learning. It is an optimized library for robotic applications and embedded devices that operates under fast duty cycles (e.g., $\leq$ 30 ms). RLLib has been tested and evaluated on RoboCup 3D soccer simulation agents,  physical NAO V4 humanoid robots, and Tiva C series launchpad microcontrollers  to predict, control, learn behaviors, and represent learnable knowledge. 
\end{itemize}}

\cventry{Jan~2014--Jul~2015}{$\mu$Energia}{C\texttt{++} Framework to Develop Embedded 
Software}{}{}{\begin{itemize} 
\item  This is a software development platform for MSP-EXP430G2 LaunchPad, Tiva C Series
EK-TM4C123GXL LaunchPad, and Tiva C Series TM4C129 Connected LaunchPad. The framework is
lightweight, flexible, and consumes minimum memory and computational resources to build
applications and rational agents on microcontrollers that sense and actuate using add-on boards.   
\end{itemize}}


\section{Languages}
\cvitemwithcomment{Sinhalese}{Native}{}
\cvitemwithcomment{English}{Fluent, written and orally}{}
%\cvitemwithcomment{Language 2}{Skill level}{Comment}
%\cvitemwithcomment{Language 3}{Skill level}{Comment}


\section{Computer Skills}
%\cvdoubleitem{Professional}{C, C\texttt{++}, Java, and Matlab software developer in Linux and Windows.}{}{}
%\cvdoubleitem{category 2}{XXX, YYY, ZZZ}{category 5}{XXX, YYY, ZZZ}
%\cvdoubleitem{category 3}{XXX, YYY, ZZZ}{category 6}{XXX, YYY, ZZZ}
\cvitem{Professional}{C, C\texttt{++}, Java, JavaScript, and Matlab for both Linux and Windows.}
\cvitem{Platform}{CUDA, Apache Hadoop, WebGL, Arduino, and Energia.}
\cvitem{IDE}{Eclipse CDT and IntelliJ IDEA.}

\section{Awards}
\cventry{}{Academic}{}{}{}{\begin{itemize} 
		\item Outstanding research assistant (2015).
		\item Outstadning teaching assistant (2010).		 
	\end{itemize}}
	\cventry{}{International Competitions: RoboCup}{}{}{}{\begin{itemize} 
			\item 1$^{\mbox{st}}$ in U.S. Open 2015.
			\item 2$^{\mbox{nd}}$ in World Cup, 2$^{\mbox{nd}}$ in U.S. Open, and  3$^{\mbox{rd}}$ in Asian Open 2014.
			\item 2$^{\mbox{nd}}$ in the World Cup, and 2$^{\mbox{nd}}$ Asian Open 2012.
			\item 1$^{\mbox{st}}$ in Europe Open 2011.
		\end{itemize}}


\section{Interests}
\cvitem{Badminton}{I am an active Badminton player since 1995. I was the president of University of 
Miami Badminton Club from 2009--2012. My duties were to manage all activities related to the club,  
and participating in tournaments. }
\cvitem{Cricket}{I professionally played Cricket for my high school, De Mazenod College, Kandana, 
Sri Lanka, from 1993--1996. Currently, I 
participate in recreational Cricket tournaments in South Florida.}
%\cvitem{hobby 2}{Description}
%\cvitem{hobby 3}{Description}

\section{Continuing Education}
\cvitem{}{In my free time, I have tried to keep my knowledge sharp in a number of areas using open 
source courses from:}
\cvitem{coursera.org}{Machine Learning (Stanford University), Neural Networks for Machine 
Learning (University of Toronto), Heterogeneous Parallel Programming (University of Illinois at 
Urbana-Champaign), Control of Mobile Robots (Georgia Institute of Technology), and Game Theory 
(Stanford University \& The University of British Columbia).}
\cvitem{edx.org}{EE40LX Electronic Interfaces (BerkeleyX).}

% \section{Extra 1}
% \cvlistitem{Item 1}
% \cvlistitem{Item 2}
% \cvlistitem{Item 3. This item is particularly long and therefore normally spans over several lines. Did you notice the indentation when the line wraps?}

% \section{Extra 2}
% \cvlistdoubleitem{Item 1}{Item 4}
% \cvlistdoubleitem{Item 2}{Item 5}
% \cvlistdoubleitem{Item 3}{Item 6. Like item 3 in the single column list before, this item is particularly long to wrap over several lines.}

% \section{References}
% \begin{cvcolumns}
%   \cvcolumn{Category 1}{\begin{itemize}\item Person 1\item Person 2\item Person 3\end{itemize}}
%   \cvcolumn{Category 2}{Amongst others:\begin{itemize}\item Person 1, and\item Person 2\end{itemize}(more upon request)}
%   \cvcolumn[0.5]{All the rest \& some more}{\textit{That} person, and \textbf{those} also (all available upon request).}
% \end{cvcolumns}

% Publications from a BibTeX file without multibib
%  for numerical labels: \renewcommand{\bibliographyitemlabel}{\@biblabel{\arabic{enumiv}}}% CONSIDER MERGING WITH PREAMBLE PART
%  to redefine the heading string ("Publications"): \renewcommand{\refname}{Articles}
%\nocite{*}
%\bibliographystyle{plain}
%\bibliography{publications}                        % 'publications' is the name of a BibTeX file

% Publications from a BibTeX file using the multibib package
%\section{Publications}
%\nocitebook{book1,book2}
%\bibliographystylebook{plain}
%\bibliographybook{publications}                   % 'publications' is the name of a BibTeX file
%\nocitemisc{misc1,misc2,misc3}
%\bibliographystylemisc{plain}
%\bibliographymisc{publications}                   % 'publications' is the name of a BibTeX file

\clearpage
%-----       letter       ---------------------------------------------------------
% recipient data
% \recipient{Company Recruitment team}{Company, Inc.\\123 somestreet\\some city}
% \date{January 01, 1984}
% \opening{Dear Sir or Madam,}
% \closing{Yours faithfully,}
% \enclosure[Attached]{curriculum vit\ae{}}          % use an optional argument to use a string other than "Enclosure", or redefine \enclname
% \makelettertitle

% Lorem ipsum dolor sit amet, consectetur adipiscing elit. Duis ullamcorper neque sit amet lectus facilisis sed luctus nisl iaculis. Vivamus at neque arcu, sed tempor quam. Curabitur pharetra tincidunt tincidunt. Morbi volutpat feugiat mauris, quis tempor neque vehicula volutpat. Duis tristique justo vel massa fermentum accumsan. Mauris ante elit, feugiat vestibulum tempor eget, eleifend ac ipsum. Donec scelerisque lobortis ipsum eu vestibulum. Pellentesque vel massa at felis accumsan rhoncus.

% Suspendisse commodo, massa eu congue tincidunt, elit mauris pellentesque orci, cursus tempor odio nisl euismod augue. Aliquam adipiscing nibh ut odio sodales et pulvinar tortor laoreet. Mauris a accumsan ligula. Class aptent taciti sociosqu ad litora torquent per conubia nostra, per inceptos himenaeos. Suspendisse vulputate sem vehicula ipsum varius nec tempus dui dapibus. Phasellus et est urna, ut auctor erat. Sed tincidunt odio id odio aliquam mattis. Donec sapien nulla, feugiat eget adipiscing sit amet, lacinia ut dolor. Phasellus tincidunt, leo a fringilla consectetur, felis diam aliquam urna, vitae aliquet lectus orci nec velit. Vivamus dapibus varius blandit.

% Duis sit amet magna ante, at sodales diam. Aenean consectetur porta risus et sagittis. Ut interdum, enim varius pellentesque tincidunt, magna libero sodales tortor, ut fermentum nunc metus a ante. Vivamus odio leo, tincidunt eu luctus ut, sollicitudin sit amet metus. Nunc sed orci lectus. Ut sodales magna sed velit volutpat sit amet pulvinar diam venenatis.

% Albert Einstein discovered that $e=mc^2$ in 1905.

% \[ e=\lim_{n \to \infty} \left(1+\frac{1}{n}\right)^n \]

% \makeletterclosing

%\clearpage\end{CJK*}                              % if you are typesetting your resume in Chinese using CJK; the \clearpage is required for fancyhdr to work correctly with CJK, though it kills the page numbering by making \lastpage undefined
\end{document}


%% end of file `template.tex'.
